%\documentclass{article}

%\usepackage[utf8]{inputenc}
%\usepackage[T1]{fontenc}      
%\usepackage[francais]{babel}
%\usepackage{graphicx}
%\usepackage{circuitikz}
%\usepackage[squaren, Gray]{SIunits}
%\usepackage{sistyle}
%\usepackage[autolanguage]{numprint}
%\usepackage{pgfplots}
%\usepackage{amsmath,amssymb,array}
%\usepackage{url}
%\usepackage[version=3]{mhchem}
%\usepackage{array} 

%\begin{document}
%%%%%% titlepage.tex %%%%%%


%\begin{titlepage}
%
%\begin{center}
%
%\textsc{\Large Université Catholique de Louvain}\\[0.5cm]
%\textsc{\Large \'Ecole Polytechnique de Louvain}\\[1cm]
%
%{\Huge \bfseries Projet}\\[0.15cm]
%
%%%\begin{center}
%%%\includegraphics[width = 8cm]{./titlepage/BARRA.jpg}
%%%\end{center}
%
%\rule{\linewidth}{0.3mm}\\[0.1cm]
%{\huge \bfseries Rapport de tache 1}\\[0.02cm]
%\rule{\linewidth}{0.3mm}\\[0.5cm]
%
%{\LARGE \bsc{Groupe} 1}\\[1cm]
%
%\begin{minipage}{0.5\textwidth}
%\begin{flushleft} \Large
%\emph{Auteurs:}\\
%\Large Simon \textsc{Boigelot} \normalsize 75971300\\
%\Large Virgile \textsc{Goyens} \normalsize 83391300\\
%\Large Corentin \textsc{Joachim} \normalsize 75971300\\
%\Large Xavier \textsc{Lambein} \normalsize 54621300\\
%\Large Edward \textsc{Nicol} \normalsize 27101300\\
%\Large Léa \textsc{Paulus} \normalsize 48251100\\
%\Large Abbas \textsc{Sliti} \normalsize 75971300
%\end{flushleft}
%\end{minipage}
%\begin{minipage}{0.4 \textwidth}
%\begin{flushright} \large
%\emph{Cours:} \\
%LFSAB1503\\
%\emph{Professeurs responsables:} \\
%J. De Wilde \\
%P. Luis Alconero \\
%D. Mignon \\
%\emph{Assistant:} \\
%*** \\
%\end{flushright}
%\end{minipage}
%
%\vfill
%
%%%\begin{minipage}{0.3\textwidth}
%%%\begin{flushleft}
%%%\includegraphics[height=2.5cm]{./titlepage/logo-ucl.jpg}
%%%\end{flushleft}
%%%\end{minipage}
%\begin{minipage}{0.3\textwidth}
%\begin{center}
%{\large FSA12BA}\\
%{\large 21 septembre 2014}
%\end{center}
%\end{minipage}
%%%\begin{minipage}{0.3\textwidth}
%%%\begin{flushright}
%%%\includegraphics[height=1cm]{./titlepage/logo-epl.jpg}
%%%\end{flushright}
%%%\end{minipage}
%
%\end{center}
%
%\end{titlepage}


\section{Bilan de masse}

L'équation de la production d'ammoniac selon le procédé Haber-Bosch est la suivante: 

$$\ce {N2 + 3H2 -> 2NH3}$$

Nous voulons calculer la quantité de réactifs pour obtenir une quantité de \unit{1000}{\tonne}  d'ammoniac. Pour cela avec les masses molaires respectives de \unit{28}{\gram/\mole} et \unit{2}{\gram/\mole} du diazote et du dihydrogène, il nous faut: 
\begin{center}
 \begin{tabular}{|l|c|r|}
   \hline
    & dihydrogène & diazote \\
   \hline
   Nb mole & $88.2\cdot 10^6$ & $29.4\cdot 10^6$ \\
   Poids & \unit{176.8}{\tonne} & \unit{823.2}{\tonne}  \\
   \hline
 \end{tabular}
\end{center}

Notre bilan de masse est maintenant fini.

\section{Aspect thermique}

Nous devons faire en sorte que notre réacteur soit à une température constante de \unit{500}{\celsius}. Étant en présence d'une réaction exothermique, pour le refroidir, nous disposons d'un flux d'eau dont la température varie entre \unit{25}{\celsius} et \unit{90}{\celsius}.


Nous utilisons comme hypothèses que le réacteur est, au départ, à \unit{500}{\celsius} et que la réaction se fait en continu.


On sait que réaction dégage \unit{46.11}{kJ/\mole}, on considère que l'eau est injectée à \unit{25}{\celsius} et qu'une fois qu'elle atteint la température de \unit{90}{\celsius} elle ressort du circuit. 
Il nous est donné dans des tables qu'à l'état liquide, l'eau a une capacité calorifique \unit{75.29}{\joule/\kelvin \cdot \mole}. Ayant une différence de température de \unit{65}{\celsius} entre l'entrée et la sortie nous obtenons l'expression suivante:
$$ 75.29\cdot 65 $$% pas complet
il nous faut \unit{9972000}{\litre/jour} pour refroidir la production de \unit{1000}{\tonne} d'ammoniac. Cela est équivalent à un débit de \unit{115}{\litre/\second}.
% petit souci, je suis pas sur que notre enthalpie de réaction soit juste :(


\section{Provenance des réactifs}
 
Pour ce qui est de la provenance des réactifs plusieurs options ont été considérées. Pour ce qui est de l'eau nous pensions utiliser l'électrolyse de l'eau. La possibilité de la décomposition thermochimique a elle été abandonné. Pour le diazote nous avons décidé de prendre le procédé Lindé qui consiste en une distillation de l'air liquide.




%\end{document}
